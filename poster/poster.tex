%\title{LaTeX Portrait Poster Template}
%%%%%%%%%%%%%%%%%%%%%%%%%%%%%%%%%%%%%%%%%
% a0poster Portrait Poster
% LaTeX Template
% Version 1.0 (22/06/13)
%
% The a0poster class was created by:
% Gerlinde Kettl and Matthias Weiser (tex@kettl.de)
%
% This template has been downloaded from:
% http://www.LaTeXTemplates.com
%
% License:
% CC BY-NC-SA 3.0 (http://creativecommons.org/licenses/by-nc-sa/3.0/)
%
%%%%%%%%%%%%%%%%%%%%%%%%%%%%%%%%%%%%%%%%%

%----------------------------------------------------------------------------------------
%   PACKAGES AND OTHER DOCUMENT CONFIGURATIONS
%----------------------------------------------------------------------------------------

\documentclass[a0,portrait]{a0poster}


\usepackage{multicol} % This is so we can have multiple columns of text side-by-side
\columnsep=100pt % This is the amount of white space between the columns in the poster
\columnseprule=3pt % This is the thickness of the black line between the columns in the poster

\usepackage[svgnames]{xcolor} % Specify colors by their 'svgnames', for a full list of all colors available see here: http://www.latextemplates.com/svgnames-colors

\usepackage{times} % Use the times font
%\usepackage{palatino} % Uncomment to use the Palatino font

\usepackage{graphicx} % Required for including images
\graphicspath{{figures/}} % Location of the graphics files
\usepackage{booktabs} % Top and bottom rules for table
\usepackage[font=small,labelfont=bf]{caption} % Required for specifying captions to tables and figures
\usepackage{amsfonts, amsmath, amsthm, amssymb} % For math fonts, symbols and environments
\usepackage{wrapfig} % Allows wrapping text around tables and figures

% Timeline
\usepackage{colortbl}
\DeclareCaptionFont{blue}{\color{openaccess}}
\newcommand{\foo}{\color{openaccess}\makebox[0pt]{\Large\textbullet}\hskip-0.5pt\vrule width 1pt\hspace{1em}}
\renewcommand\arraystretch{1.4}\arrayrulecolor{openaccess}

\usepackage[utf8x]{inputenc} %tildes
\usepackage[T1]{fontenc}
\usepackage{tikz}

\usepackage{fontspec}
\setsansfont{Fira Sans}
\setmainfont{Crimson Text}

\usepackage{sectsty}
\allsectionsfont{\sffamily}

\definecolor{openaccess}{RGB}{246,141,10}
\definecolor{oldgold}{rgb}{0.81, 0.71, 0.23}
\definecolor{palegreen}{rgb}{0.6, 0.98, 0.6}
\definecolor{gray}{rgb}{0.5, 0.5, 0.5}
\definecolor{kellygreen}{rgb}{0.3, 0.73, 0.09}

\newcommand{\tikzcircle}[2][red,fill=red]{\tikz[baseline=-0.5ex]\draw[#1,radius=#2] (0,0) circle ;}%

\usepackage{wrapfig}

\begin{document}

%----------------------------------------------------------------------------------------
%   POSTER HEADER
%----------------------------------------------------------------------------------------

% The header is divided into two boxes:
% The first is 75% wide and houses the title, subtitle, names, university/organization and contact information
% The second is 25% wide and houses a logo for your university/organization or a photo of you
% The widths of these boxes can be easily edited to accommodate your content as you see fit

\begin{minipage}[t]{0.7\linewidth}
  \vspace{0pt}
  \flushleft
  \VeryHuge \color{openaccess} \textbf{\sffamily Open Access: revisión histórica y situación actual} \color{Black}
  \\ % Title
  %\Huge\textit{Country Update}\\[2.4cm] % Subtitle

  \vspace{1cm} % A bit of extra whitespace between the header and poster content
  \large \textbf{Cristina Heredia, David Charte, Alejandro Alcalde}\\[0.5cm] % Author(s)
  \large Departamento de Ciencias de la Computación e Inteligencia Artificial, Universidad de Granada\\[0.5cm] % advisor
%  \large \textbf{Alejandro Alcalde} \\[0.5cm] % University/organization
\end{minipage}
\hfill
\begin{minipage}[t]{0.15\textwidth}
  \vspace{0.5em}
  \centering\includegraphics[width=0.60\textwidth]{openaccess_logo.png}
\end{minipage}
%
\vspace{1cm} % A bit of extra whitespace between the header and poster content

%----------------------------------------------------------------------------------------

\begin{multicols}{2} % This is how many columns your poster will be broken into, a portrait poster is generally split into 2 columns

  %----------------------------------------------------------------------------------------
  %   ABSTRACT
  %----------------------------------------------------------------------------------------

  %\color{Navy} % Navy color for the abstract

  %----------------------------------------------------------------------------------------
  %   Introduction
  %----------------------------------------------------------------------------------------
\color{Black} % SaddleBrown color for the introduction
  \section*{Introducción}
  \begin{itemize}
  \item En 1665 aparecen primeras revistas, a los autores no se les pagaba por publicar en ellas.
  \item Ya que éstas se hicieron populares, los autores siguieron publicando en ellas, por impacto más que por dinero.
  \item \textbf{Problema:} Los precios comenzaron a subir hasta hacerse inasequibles. Hasta la aparición de internet como alternativa. Sin embargo, las revistas siguen cobrando demasiado en la actualizad, a pesar de no tener a penas gastos de publicación y edición.
  \end{itemize}

  \section*{Open Access (OA)}
  \begin{itemize}
  \item Tras la llegada de internet, nació \textbf{Open Access}, una forma de \textbf{acceder a la información}, no un \textbf{modelo de negocio}. Permite acceder a la información de \textbf{forma gratuita}. NO es incompatible con el uso de CopyRight.
  \item Alternativa al modelo tradicional, donde los autores a veces \textbf{pagan} por publicar y los lectores también deben \textbf{pagar} por acceder a las publicaciones.
  \item \textit{Formas de fomentar OA}
    \begin{itemize}
    \item Dejar una copia del artículo en un repositorio OA (\textbf{arXiv}).
    \item Publicar en revistas OA (\textbf{BioMed, Public Library of Science}).
    \item Subir una copia del artículo original a una web personal.
    \item Publicar en revistas híbridas, donde el autor paga para hacer el artículo OA.
    \end{itemize}
  \item Las revistas OA siguen un código de colores:
    \begin{itemize}
    \item[] \tikzcircle[oldgold, fill=oldgold]{50pt} Acceso libre sin retraso.
    \item[] \tikzcircle[kellygreen, fill=kellygreen]{50pt} Permite el archivo de trabajos ya publicados.
    \item[] \tikzcircle[palegreen, fill=palegreen]{50pt} Permite el archivo de trabajos que se han mandado a revisión pero aún no publicados.
    \item[] \tikzcircle[gray, fill=gray]{50pt} Indica que no cumple ninguna de las anteriores.
    \end{itemize}
  \end{itemize}


 %----------------------------------------------------------------------------------------
  %   OBJECTIVES
  %----------------------------------------------------------------------------------------

  \color{Black} % SaddleBrown color for the introduction
  \section*{Modelo tradicional: problemática y escándalos}
  %
  \begin{itemize}
  \item $\downarrow$ Coste de publicación, maquetación y distribución VS $\uparrow$ coste de revistas
  \item Bases de la carrera investigadora:
    \begin{itemize}
    \item publicacionesde calidad 
    \item en revistas de calidad
    \end{itemize}
  \item Los investigadores revisan de forma voluntaria
  \item Editoriales sacan beneficios elevados aprovechando el voluntariado (Elsevier, Springer) 
  \item Boicot a Elsevier por varios investigadores
  \item Coste de la revista (precio por pág)
 \begin{itemize}
    \item The Annals of Mathematics- 0.13/pág
    \item Elsevier -1.30/pág
    \end{itemize}
  \item Escándalos de Elsevier:
    \begin{itemize}
  	\item citaciones mutuas de Elsevier con  Chaos, Solitions \& Fractals
  	\item recibió dinero de farmaceuticas para publicar artículos
  	\end{itemize}
  \end{itemize}

%  \begin{figure}[p!]
  %   \centering
  \section*{Timeline del Open Access}
  %\includegraphics[width=0.9\columnwidth]{timeline.pdf}
  \begin{tabular}{@{\,}r <{\hskip 1em} !{\foo} >{\raggedright\arraybackslash}p{20em}}
%\toprule
  \addlinespace[1.5ex]
  1665 & Primeras revistas científicas (Londres, París)\vspace{5em}\\
  1987 & Primera revista Open Access online: \textit{New Horizons in Adult Education}\vspace{5em}\\
  2002 & Publicación de la \textit{Budapest Open Access Initiative}\\
  2002 & Creación del \textit{Directory of Open Access Journals}\vspace{5em}\\
  2008 & Publicación del \textit{Guerilla Open Access Manifesto} (Aaron Swartz)\vspace{2em}\\
  2011 & Creación de Sci-Hub (Alexandra Elbakyan)\\
  2012 & Boicot a Elsevier: \textit{The Cost of Knowledge}\\
  2012 & Legislación del Reino Unido para liberar la investigación del Department for International Development\\
  2013 & Estados Unidos aprueba el \textit{Fair Access to Science and Technology Research Act}\vspace{1em}\\
  2015 & Boicot a  \textit{Lingua} de su equipo editorial\\
  2016 & Llamada a la acción de Amsterdan para la ciencia abierta en Europa\\
  2017 & Más de 10000 revistas Open Access en el DOAJ\vspace{2em}\\
  2020 & Todas las publicaciones científicas financiadas públicamente en la UE podrían ser de libre acceso\\
  \end{tabular}

  %  \caption{Una caption}
  %\end{figure}


\end{multicols}
\end{document}
