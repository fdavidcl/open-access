%\title{LaTeX Portrait Poster Template}
%%%%%%%%%%%%%%%%%%%%%%%%%%%%%%%%%%%%%%%%%
% a0poster Portrait Poster
% LaTeX Template
% Version 1.0 (22/06/13)
%
% The a0poster class was created by:
% Gerlinde Kettl and Matthias Weiser (tex@kettl.de)
%
% This template has been downloaded from:
% http://www.LaTeXTemplates.com
%
% License:
% CC BY-NC-SA 3.0 (http://creativecommons.org/licenses/by-nc-sa/3.0/)
%
%%%%%%%%%%%%%%%%%%%%%%%%%%%%%%%%%%%%%%%%%

%----------------------------------------------------------------------------------------
%   PACKAGES AND OTHER DOCUMENT CONFIGURATIONS
%----------------------------------------------------------------------------------------

\documentclass[a0,portrait]{a0poster}
\usepackage[fontsize=38]{scrextend}

\usepackage{multicol} % This is so we can have multiple columns of text side-by-side
\columnsep=6em % This is the amount of white space between the columns in the poster
\columnseprule=0.5pt % This is the thickness of the black line between the columns in the poster

\usepackage[svgnames]{xcolor} % Specify colors by their 'svgnames', for a full list of all colors available see here: http://www.latextemplates.com/svgnames-colors

\usepackage{times} % Use the times font
%\usepackage{palatino} % Uncomment to use the Palatino font

\usepackage{graphicx} % Required for including images
\graphicspath{{figures/}} % Location of the graphics files
\usepackage{booktabs} % Top and bottom rules for table
\usepackage[font=small,labelfont=bf]{caption} % Required for specifying captions to tables and figures
\usepackage{amsfonts, amsmath, amsthm, amssymb} % For math fonts, symbols and environments
\usepackage{wrapfig} % Allows wrapping text around tables and figures
\usepackage{hyperref}
%\setlength{\parskip}{10cm}

% Timeline
\usepackage{colortbl}
\DeclareCaptionFont{blue}{\color{openaccess}}
\newcommand{\foo}{\color{openaccess}\makebox[0pt]{\Large\textbullet}\hskip-0.5pt\vrule width 1pt\hspace{1em}}
\renewcommand\arraystretch{1.4}\arrayrulecolor{openaccess}

\usepackage[utf8x]{inputenc} %tildes
\usepackage[T1]{fontenc}
\usepackage{tikz}

\usepackage{fontspec}
\setsansfont{Fira Sans}
\setmainfont{Crimson Text}
\usepackage[most]{tcolorbox}

\usepackage{sectsty}
\allsectionsfont{\sffamily}

\definecolor{openaccess}{RGB}{246,141,10}
\definecolor{oldgold}{rgb}{0.81, 0.71, 0.23}
\definecolor{palegreen}{rgb}{0.6, 0.98, 0.6}
\definecolor{gray}{rgb}{0.5, 0.5, 0.5}
\definecolor{kellygreen}{rgb}{0.3, 0.73, 0.09}

\newcommand{\tikzcircle}[2][red,fill=red]{\tikz[baseline=-1.5ex]\draw[#1,radius=#2] (0,0) circle ;}%

\begin{document}

%----------------------------------------------------------------------------------------
%   POSTER HEADER
%----------------------------------------------------------------------------------------

% The header is divided into two boxes:
% The first is 75% wide and houses the title, subtitle, names, university/organization and contact information
% The second is 25% wide and houses a logo for your university/organization or a photo of you
% The widths of these boxes can be easily edited to accommodate your content as you see fit

\begin{minipage}[t]{0.7\linewidth}
  \vspace{0pt}
  \flushleft
  \VeryHuge \color{openaccess} \textbf{\sffamily Open Access: revisión histórica y situación actual} \color{Black}
  \\ % Title
  %\Huge\textit{Country Update}\\[2.4cm] % Subtitle

  \vspace{1cm} % A bit of extra whitespace between the header and poster content
  \large \textbf{Cristina Heredia, David Charte, Alejandro Alcalde}\\[0.5cm] % Author(s)
  \large Departamento de Ciencias de la Computación e Inteligencia Artificial, Universidad de Granada\\[0.5cm] % advisor
%  \large \textbf{Alejandro Alcalde} \\[0.5cm] % University/organization
\end{minipage}
\hfill
\begin{minipage}[t]{0.15\textwidth}
  \vspace{0.5em}
  \centering\includegraphics[width=0.60\textwidth]{openaccess_logo.png}
\end{minipage}
%
\vspace{1cm} % A bit of extra whitespace between the header and poster content

%----------------------------------------------------------------------------------------

\begin{multicols}{2} % This is how many columns your poster will be broken into, a portrait poster is generally split into 2 columns

  %----------------------------------------------------------------------------------------
  %   ABSTRACT
  %----------------------------------------------------------------------------------------

  %\color{Navy} % Navy color for the abstract

  %----------------------------------------------------------------------------------------
  %   Introduction
  %----------------------------------------------------------------------------------------
\color{Black} % SaddleBrown color for the introduction
\section*{Introducción}

\begin{tcolorbox}[colback=openaccess!5!white,colframe=openaccess!75!black,boxsep=1cm]
  \setlength{\parskip}{0.5cm}
  
  En 1665 aparecen primeras revistas, a los autores no se les pagaba por publicar en ellas.

  Ya que éstas se hicieron populares, los autores siguieron publicando en ellas, por impacto más que por dinero.

  \textbf{Problema:} Los precios comenzaron a subir hasta hacerse inasequibles. Hasta la aparición de internet como alternativa. Sin embargo, las revistas siguen cobrando demasiado en la actualizad, a pesar de no tener a penas gastos de publicación y edición.
  
\end{tcolorbox}

\section*{Open Access (OA)}
\begin{tcolorbox}[colback=openaccess!5!white,colframe=openaccess!75!black,boxsep=1cm]
  \setlength{\parskip}{0.5cm}
  
\textbf{Open Access} es una forma de \textbf{acceder a la información}, no un modelo de negocio. Permite \textbf{acceso gratuito}. Compatible con el uso de copyright. Alternativa al modelo tradicional, donde los lectores también deben pagar por acceder a las publicaciones.

Formas de fomentar OA
    \begin{itemize}
    \item Dejar una copia del artículo en un repositorio OA (\textbf{arXiv}).
    \item Publicar en revistas OA (\textbf{BioMed, Public Library of Science}).
    \item Subir una copia del artículo original a una web personal.
    \item Publicar en revistas híbridas, donde el autor paga para hacer el artículo OA.
    \end{itemize}

    Las revistas OA siguen un código de colores:
    \begin{itemize}
    \item[] \tikzcircle[oldgold, fill=oldgold]{25pt}\quad Acceso libre sin retraso
    \item[] \tikzcircle[kellygreen, fill=kellygreen]{25pt}\quad Archivado de trabajos ya publicados
    \item[] \tikzcircle[palegreen, fill=palegreen]{25pt}\quad Archivado de trabajos en revisión
    \item[] \tikzcircle[gray, fill=gray]{25pt}\quad No cumple ninguna de las anteriores
    \end{itemize}
\end{tcolorbox}


 %----------------------------------------------------------------------------------------
  %   OBJECTIVES
  %----------------------------------------------------------------------------------------

  \color{Black} % SaddleBrown color for the introduction
  \section*{Modelo tradicional}
  \begin{tcolorbox}[colback=openaccess!5!white,colframe=openaccess!75!black,boxsep=1cm,breakable,pad at break*=1cm]
    \setlength{\parskip}{0.5cm}
    %breakable
    
    $\downarrow$ Coste de publicación, maquetación y distribución VS $\uparrow$ coste de revistas.
    
    Bases de la carrera investigadora: publicaciones de calidad en revistas de calidad.
    
    Los investigadores revisan de forma voluntaria. Editoriales sacan beneficios elevados aprovechando el voluntariado (Elsevier, Springer).

    Coste de la revista (precio por pág)
    \begin{itemize}
    \item The Annals of Mathematics: 0.13/pág
    \item Elsevier: 1.30/pág
    \end{itemize}

    Escándalos de Elsevier:
    \begin{itemize}
    \item Citaciones mutuas de Elsevier con  Chaos, Solitions \& Fractals
    \item Recibió dinero de farmacéuticas para publicar artículos
    \end{itemize}
  \end{tcolorbox}
%  \begin{figure}[p!]
  %   \centering
  \section*{Timeline del Open Access}
  %\includegraphics[width=0.9\columnwidth]{timeline.pdf}
  \begin{tabular}{@{\,}r <{\hskip 1em} !{\foo} >{\raggedright\arraybackslash}p{20em}}
%\toprule
  \addlinespace[1.5ex]
  1665 & Primeras revistas científicas (Londres, París)\vspace{5em}\\
  1987 & Primera revista Open Access online: \textit{New Horizons in Adult Education}\vspace{5em}\\
  2002 & Publicación de la \textit{Budapest Open Access Initiative}\\
  2002 & Creación del \textit{Directory of Open Access Journals}\vspace{5em}\\
  2008 & Publicación del \textit{Guerilla Open Access Manifesto} (Aaron Swartz)\vspace{2em}\\
  2011 & Creación de Sci-Hub (Alexandra Elbakyan)\\
  2012 & Boicot a Elsevier: \textit{The Cost of Knowledge}\\
  2012 & Legislación del Reino Unido para liberar la investigación del Department for International Development\\
  2013 & Estados Unidos aprueba el \textit{Fair Access to Science and Technology Research Act}\vspace{1em}\\
  2015 & Boicot a  \textit{Lingua} de su equipo editorial\\
  2016 & Llamada a la acción de Amsterdan para la ciencia abierta en Europa\\
  2017 & Más de 10000 revistas Open Access en el DOAJ\vspace{2em}\\
  2020 & Todas las publicaciones científicas financiadas públicamente en la UE podrían ser de libre acceso\\
  \end{tabular}

  %  \caption{Una caption}
  %\end{figure}

  \flushleft
  \includegraphics[width=0.8\columnwidth]{../doaj_years.png}

  {\small Total de OAJs listados en el \textit{Directory of Open Access Journals} \\ desde su creación en 2002. Datos de \url{https://doaj.org/csv}}

\end{multicols}
\end{document}
