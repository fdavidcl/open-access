%
% PARA COMPILARLO Y CREAR EL PDF
%    pdflatex paper.tex && pdflatex paper.tex && pdflatex paper.tex 
%

\documentclass{llncs}
\usepackage{graphics}
\usepackage[dvips]{epsfig}
\usepackage[utf8]{inputenc} %tildes
%\usepackage[latin1]{inputenc} % tildes

\def\CC{{C\hspace{-.05em}\raisebox{.4ex}{\tiny\bf ++}}~}
\addtolength{\textfloatsep}{-0.5cm}
\addtolength{\intextsep}{-0.5cm}


%%%%%%%%%%%%%%%% Título %%%%%%%%%%%%%%%
\title{El título del documento}

\author { Nombre Apellido \inst{1} }

\institute{
           Departamento \\
           Universidad \\
           Direcc. \\
           e-mail: {\tt email@dominio.com}
}

\date{} 

\begin{document}

\maketitle


%%%%%%%%%%%%%%%% abstract %%%%%%%%%%%%%%%
\begin{abstract}

El texto del resumen de este documento. El texto del resumen de este documento. El texto del resumen de este documento. El texto del resumen de este documento. El texto del resumen de este documento. El texto del resumen de este documento. El texto del resumen de este documento. El texto del resumen de este documento. El texto del resumen de este documento. El texto del resumen de este documento. El texto del resumen de este documento. El texto del resumen de este documento. 

\end{abstract}



%%%%%%%%%%%%%%%% Introduction %%%%%%%%%%%%%%%
\section{Primera Sección}
\label{sec:uno}

El texto de la primera sección. Ahora \emph{texto en cursiva}. El texto de la primera sección. El texto de la primera sección. El texto de la primera sección. El texto de la primera sección. 

El texto de la primera sección. El texto de la primera sección Y una referencia bibliográfica a un libro \cite{Rumelhart}.

El texto de la primera sección. El texto de la primera sección Y una referencia bibliográfica a un artículo \cite{heli1}.

Una enumeración:

\begin{enumerate}
	\item un item.
	\item otro item.
	\item tercer item.
\end{enumerate}


%%%%%%%%%%%%%%%% Method %%%%%%%%%%%%%%%
\section{Segunda Sección}
\label{sec:dos}

El texto de la segunda sección. El texto de la segunda sección. El texto de la segunda sección. El texto de la segunda sección. El texto de la segunda sección. Ahora \textbf{texto en negrita}. El texto de la segunda sección. El texto de la segunda sección. El texto de la segunda sección. 

El texto de la segunda sección. El texto de la segunda sección Y referencia a la Sección \ref{sec:uno}.

Vamos a incluir la Figura \ref{fig:image} (es un archivo en formato .EPS)

\begin{figure}
\begin{center}
\includegraphics{image.eps} \\
\caption{\small{El texto del pie de figura.}}
\label{fig:image}
\end{center}
\end{figure}

Y finalmente, mostramos la Tabla \ref{table:tabla}.

\begin{table}
\begin{center}
\begin{tabular}{|c|c|c|}
\hline 
título 1 & título 2 & título 3  \\
\hline
\hline
columna 1 & columna 2 & columna 3  \\
\hline
columna 1 & columna 2 & columna 3  \\
\hline
columna 1 & columna 2 & columna 3  \\
\hline
columna 1 & columna 2 & columna 3  \\
\hline
\end{tabular}
\end{center}
\caption{\small{El texto del pie de tabla.}}
\label{table:tabla}
\end{table}


%%%%%%%%%%%%%% Bibliografy %%%%%%%%%%%%%%%
\bibliographystyle{plain}
\bibliography{refs}

\end{document}
