\section{Introducción}
\section{Amenazas y desafios}
\section{Soluciones e iniciativas}

Desde la publicación del manifiesto de la Budapest OA Initiative \cite{boai} en 2002, han surgido distintas iniciativas que tratan de promover la publicación científica en acceso abierto. Estos estímulos proceden de distintas fuentes, tanto del ámbito gubernamental como el editorial, e incluso de personas concretas cuyos proyectos han afectado al curso del acceso libre a la ciencia.

\subsection{Iniciativas investigadoras}

Diferentes editoriales han puesto en marcha la creación de revistas de acceso abierto (\textit{Open Access Journals, OAJ}). El primer OAJ fue \textit{New Horizons in Adult Education} de la editorial Wiley \cite{earlyoaj}, fundado en 1987. Le siguieron \textit{Psycoloquy} y \textit{Public Access-Computer Systems Review} en 1989. A partir de 1990, fueron surgiendo más OAJs de forma incremental. En la actualidad, el \textit{Directory of Open Access Journals}\footnote{Disponible en \url{https://doaj.org/}, accedido el 29 de noviembre de 2017.} contabiliza un total de 10544 revistas de acceso abierto.

Glossa vs Lingua.

- BioMed Central
- Nucleic Acids Research (OUP)
- Molecular Systems Biology (Nature)

\subsection{Legislación a favor de Open Access}

Holanda, Noruega, UE \cite{enserink2016dramatic}

\subsection{Iniciativas independientes}

Swartz, Sci-Hub

\section{Conclusiones}

Antes de la era de la información y tecnología, el rol de las revistas científicas era múltiple. El principal era la divulgación de las investigaciones científicas. Las revistas cobraban el coste de la maquetación, el cual no era sencillo para matemáticas, el coste de publicar en formato físico y el de distribución de los mismos a los suscriptores. En la era digital, nada de esto supone un coste tan elevado.

https://www.theguardian.com/science/2017/jun/27/profitable-business-scientific-publishing-bad-for-science
